\documentclass{article}
\usepackage{graphicx}
\usepackage{hyperref}
\usepackage{xcolor}
\usepackage{amsmath}
\usepackage{bm}
\usepackage{commath}

%\usepackage{siunitx}
\usepackage[margin=0.5in]{geometry}
\definecolor{darkgreen}{rgb}{0.0, 0.6, 0.0}
\usepackage[parfill]{parskip}

\bibliographystyle{acm}

\begin{document}

\title{A synthetic multicellular system to deliver a therapeutic "cargo"}
\date{}
\author{}
\maketitle

\section{Introduction}
This app demonstrates how simple ``rules'' for chemical- and contact-based
communication could be used for \emph{active} delivery of therapeutic
compounds, while overcoming traditional biotransport limits and using
hypoxia to target delivery. It is based on a published PhysiCell example
in Ghaffarizadeh et al.\cite{physicell-2018}. In this model, \textcolor{darkgreen}{cancer cells
(green)} rapidly divide while consuming oxygen, which drives the
emergence of hypoxic gradients. Greater availability of oxygen drives
faster proliferation, while low oxygen can trigger necrotic death,
creating a necrotic core ({brown cells}) in the center of the tumor.
After initial growth, we "inject" a multicellular synthetic therapy.

In this (for now theoretical) synthetic therapy, \textcolor{red}{"worker cells" (red)}
search for \textcolor{blue}{"cargo cells" (blue)}, form focal adhesions, and haul their
cargo towards hypoxic (oxygen-depleted) regions in the tissue. Once they
reach a region of sufficiently low oxygen, they release their cargo and
resume their initial search for more cargo. Released cargo cells secrete
a chemotherapeutic compound that can kill nearby tumor cells. (Cancer
cells are shaded according to their level of therapy-induced damage.)

This model and cloud-hosted demo are part of a course on computational
multicellular systems biology created and taught by Dr. Paul Macklin in
the Department of Intelligent Systems Engineering at Indiana University.
It is also part of the education and outreach for the NSF-funded IU Engineered
nanoBIO Node and the NCI-funded cancer systems biology grants.
The models are built using
\href{https://journals.plos.org/ploscompbiol/article?id=10.1371/journal.pcbi.1005991}{PhysiCell}:
a C++ framework for multicellular systems biology.

\section{Chemical components}
All chemical signals move by chemical diffusion in the simulated
environment, using
\href{http://dx.doi.org/10.1093/bioinformatics/btv730}{BioFVM}\cite{Ghaffarizadeh2016-ga}
to solve the reaction-diffusion equations. A key property is the
\emph{diffusion length scale}. Diffusion (with a diffusion coefficient
\emph{D}) helps spread a signal over large distances, while decay (and
uptake) (with coefficient $\lambda$) eliminates the signal to slow its
spread. These effects compete to determine the characteristic distance
\emph{L} that a chemical signal travels.

That length scale, \emph{L}, is given by \emph{L} = sqrt( \emph{D}/$\lambda$).

%Choosing \emph{D} and \emph{λ} can help to tune the chemical
Choosing \emph{D} and $\lambda$ can help to tune the chemical
communication distance in these models.

\subsection{Oxygen ($\sigma$)}
%(\emph{σ}):}{Oxygen (σ):}}\label{oxygen-ux3c3}}

This is released by the computational boundary (a simplified model of a
far-field vasculature), and consumed by tumor cells.
We use a far-field value of 38 mmHg (about 5\%), which is a typical
physioxic condition. See \cite{Ghaffarizadeh2016-ga} and \cite{physicell-2018} for more references. We use a
%length scale of 1000 μm, and cancer cell uptake gives a length scale of
length scale of 1000~$\mu$m, and cancer cell uptake gives a length scale of
100~$\mu$m  in densely packed regions.

\subsection{Chemoattractant (\emph{a}):}

This factor is released by undocked cargo cells to help guide chemotaxis
by worker cells.

The chemoattractant is nondimensionalized so that $0 \leq \emph{a} \leq 1$, and
we initially assign a length scale of 100~$\mu$m.

\subsection{Chemotherapeutic (\emph{c}):}
This factor is released by released cargo cells to kill cancer cells.

The therapeutic is nondimensionalized so that $0 \leq \emph{c} \leq 1$, and we
initially assign a length scale of 80~$\mu$m.

\section{Cell agents}
We use agent-based models to explore multicellular systems like this
one. Each cell is modeled as a software agent with an independent state,
and its own rules to change its behavior based on local environmental
conditions and communication. We use PhysiCell as our modeling platform
{[}1{]}.~

\subsection{Key features in the agent-based
models}

\paragraph{Birth and death:}\label{birth-and-death}

Cells can proliferate (divide into two daughter cells of half size),
with division rates regulated by cell rules. In our model, cells
continuously work to grow towards a "target" mature volume.

Note that birth and death are stochastic events for each cell agent: if
such a process occurs at rate \emph{r}, then between now (\emph{t}) and
soon (\emph{t} + $\Delta$\emph{t}), the probability of the event occurring for
that agent is \emph{r}$\Delta$\emph{t}.

\paragraph{Secretion and uptake:}\label{secretion-and-uptake}

Cells can secrete chemical factors, or they can remove them (i.e.,
consume or uptake). This can contribute to gradients in chemical
factors, and it's a key method of communication between cells. Moreover,
cells can "sample" the chemical state of their surrounding environment.
Secreting is sending a signal. Uptaking and sampling is receiving a
signal.

\paragraph{Biased migration:}\label{biased-migration}

Motile cells perform \emph{biased migration}. After traveling for
\emph{T}\textsubscript{persistence} time (the mean persistence time),
they choose a new migration direction $\mathbf{d}_{migrate}$. Based
on the environmental conditions, the cell chooses a directional bias
(intended direction) $\mathbf{d}_{bias}$ and a random unit vector
$\mathbf{r}$. The motile direction $\mathbf{d}_{migrate}$ is then determined
by according to:

%{d}\textsubscript{migrate} = b {d}\textsubscript{bias} + (1-b) {r}.
\begin{center}
$\mathbf{d}_{migrate} = b \: \mathbf{d}_{bias} + (1-b) \mathbf{r}$.
\end{center}

Here, \emph{b} is a \emph{bias} parameter (between 0 and 1) that
determines how strongly biased migration is towards
$\mathbf{d}_{bias}$. If \emph{b} =1, then migration is deterministic
towards $\mathbf{d}_{bias}$. If \emph{b} = 0, then migration is
completely random (Brownian).

Once the migration direction is chosen, it is normalized, and multiplied
by the cell's migration speed.

\paragraph{Mechanics:}\label{mechanics}

Cell agents can stick to one another within a prescribed interaction
distance (some multiple of their radius), and they can exert a pushing
force on neighbors. PhysiCell {[}1{]} uses potential functions to
implement these simple mechanics. Notably, PhysiCell is an off-lattice
model, meaning that cells can have variable sizes, and can move freely
without grid artifacts.

They aren't required to move some prescribed number of spatial steps.
They also can divide without checking for an open neighbor "site".
Instead, they can divide and push their neighbors out of the way.~

\paragraph{Focal adhesions:}

PhysiCell allows custom functions for any cell agents. We use this to
model focal cell-cell adhesions as spring-like forces. If two cells
\emph{i} and \emph{j} (with positions $\mathbf{x}_{i}$ and
$\mathbf{x}_{j}$) have a focal adhesion, then cell \emph{i} is
adhered to cell \emph{j} with a spring-like force that we model as:

\begin{center}
$\mathbf{F}_{ij} = \varepsilon (\mathbf{x}_{j} -
\mathbf{x}_{i})$,
\end{center}

where $\varepsilon$ is the elastic coefficient. Because PhysiCell has an
inertialess formatulation (see the method paper in {[}1{]}), this is
added directly to the cell's velocity, and so $\varepsilon$ has units
min\textsuperscript{-1}. To model the finite deformability of cells and
these elastic focal adhesions, we detach the cells if their distance
($\norm{\mathbf{x}_{i} - \mathbf{x}_{j}}$) 
exceeds a set maximum distance $r_{max}$.

\subsubsection{Worker cells (\textcolor{red}{red})}
%({red})}{Worker cells (red)}}\label{worker-cells-red}

Worker cells do not divide, but they can be assigned an apoptotic death
rate to model normal "wear and tear." We model workers as non-adhesive
except by spring-like focal adhesions; see
\protect\hyperlink{focal_adhesions}{focal adhesions}. They are made more
"repulsive" (by default, fivefold more repulsive) to allow them to more
readily move through dense cellular regions. Worker cells consume
oxygen, but at a reduced rate (by default, 10\% of the rate of cancer
cells).

{Unattached} worker cells use the following as their migration bias
direction:~

\begin{center}
$\mathbf{d}_{bias} = \nabla a$,
\end{center}

where \emph{a} is the chemoattractant released by unattached cargo
cells. Note that $\nabla$ is the gradient operator that points towards higher
concentrations of a chemical factor.

Unattached worker cells test for near contact with unattached cargo
cells. If they detect an unattached cargo cell with active receptor
\emph{R} (\emph{R} $\approx$ 1), they form a focal adhesion (see
\protect\hyperlink{focal_adhesions}{focal adhesions}).

{Attached} worker cells change their migration bias direction to:

\begin{center}
$\mathbf{d}_{bias} = - \nabla \sigma$,
\end{center}

where $\sigma$ is the oxygen.

\subsubsection{Cargo cells (\textcolor{blue}{blue})}\label{cargo-cells-blue}

Cargo cells do not divide, but they can be assigned an apoptotic death
rate to model normal "wear and tear." We model cargo as non-adhesive
except by spring-like focal adhesions; see
\protect\hyperlink{focal_adhesions}{focal adhesions}. They are made more
"repulsive" (by default, fivefold more repulsive) to allow them to more
readily move through dense cellular regions. Cargo cells consume oxygen,
but at a reduced rate (by default, 10\% of the rate of cancer cells).
Note that all cargo cells are non-motile.

Initially, all cargo cells express a receptor \emph{R} with
dimensionless value 1, which is used to control their chemical
secretions. If \emph{R} = 1, they secrete chemoattractant \emph{a} to
help guide worker cells. If \emph{R} = 0 and they are unattached, then
they secrete the therapeutic \emph{c}.

Cargo cells permanently deactivate their receptor (set \emph{R} = 0)
once they are attached to a worker cell. They detach from worker cells
if $\sigma < \sigma_{release}$.

\subsubsection{Cancer cells (\textcolor{darkgreen}{green})}\label{cancer-cells-green}

Cancer cells divide at a rate proportional to oxygen availability, and
they experience necrotic death at a rate that increases as oxygen
decreases. We use the default parameter values from {[}1{]} for this
sample model; they cannot be adjusted by users in the current demo
version. Note that all cancer cells are non-motile in this example.

Beyond oxygen-driven birth and hypoxia-driven death, cancer cells are
have a basic damage-repair model. Exposure to the chemotherapeutic
compound \emph{c} causes damage ($\omega$, which can be repaired. We
simulate this as

\begin{center}
$\frac{d\omega}{dt} = r_{damage} \: c - r_{repair} \: \omega$
\end{center}

where \emph{c} is sampled at the cell's position. Cancer cells increase
their apoptotic death rate by their therapeutic death rate
\emph{d}\textsubscript{therapy}, which we model as

\begin{center}
$d_{therapy} = \frac{\omega}{\omega_{max}} \: d_{max}$,
\end{center}

where \emph{d}\textsubscript{max} is the maximum death rate that is
reach when damage ($\omega$) reaches \emph{ω}\textsubscript{max}.

In this model, we set $\omega_{max}$ by solving the damage
model to find the steady state damage at a maximum value of \emph{c}
(which is 1):

\begin{center}
$\omega_{max} = r_{damage} / r_{repair}$
\end{center}

\subsection{Basic
instructions}\label{basic-instructions}

Modify the parameters in the "Config Basics" and "User Params" tabs
and click the "Run" button.

To view simulation results, click the "Out: Plots" tab, and slide the bar to
advance through simulation frames. Note that as the simulation runs, the
"\# frames" field will increase, so you can view more
simulation frames.

Note that you can download full simulation data for further exploration
in your tools of choice.

\section{Additional information}
To learn more about our work, please visit \href{http://mathCancer.org}{MathCancer.org}.

Credits:

Scientific Staff:
Randy Heiland, Research Associate, Intelligent Systems Engineering, Indiana University.
\begin{itemize}
\item Lead software architect:
\end{itemize}

Paul Macklin, Ph.D. , Associate Professor, Intelligent Systems Engineering, Indiana University.
\begin{itemize}
\item Scientific lead, developed PhysiCell simulation model:
\end{itemize}

The following undergraduate students have contributed to this project:

Jupyter notebook and UI development:
\begin{itemize}
\item Eric Bower (IU Intelligent Systems Engineering, B.S., 2018-present)
\item Daniel Mishler (IU Intelligent Systems Engineering, B.S., 2018-present)
\item Tyler Zhang (IU Intelligent Systems Engineering, B.S., 2018)
\end{itemize}


\section{Sponsored by}
\begin{itemize}
\item NSF EEC-1720625. Network for Computational Nanotechnology - Engineered nanoBIO Node
\item Breast Cancer Research Foundation 
\item Jayne Koskinas Ted Giovanis Foundation for Health and Policy
\item NIH U01CA232137
\end{itemize}
%\clearpage
\bibliography{refs}
\end{document}